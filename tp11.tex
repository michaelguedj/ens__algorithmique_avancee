% !TEX TS-program = pdflatex
% !TEX encoding = UTF-8 Unicode

% This is a simple template for a LaTeX document using the "article" class.
% See "book", "report", "letter" for other types of document.

\documentclass[11pt]{article} % use larger type; default would be 10pt

\usepackage[utf8]{inputenc} % set input encoding (not needed with XeLaTeX)
\usepackage[T1]{fontenc}



%%% Examples of Article customizations
% These packages are optional, depending whether you want the features they provide.
% See the LaTeX Companion or other references for full information.

%%% PAGE DIMENSIONS
\usepackage{geometry} % to change the page dimensions
\geometry{a4paper} % or letterpaper (US) or a5paper or....
% \geometry{margin=2in} % for example, change the margins to 2 inches all round
% \geometry{landscape} % set up the page for landscape
%   read geometry.pdf for detailed page layout information

\usepackage{graphicx} % support the \includegraphics command and options
\usepackage{kbordermatrix}

% \usepackage[parfill]{parskip} % Activate to begin paragraphs with an empty line rather than an indent

%%% PACKAGES
\usepackage{booktabs} % for much better looking tables
\usepackage{array} % for better arrays (eg matrices) in maths
\usepackage{paralist} % very flexible & customisable lists (eg. enumerate/itemize, etc.)
\usepackage{verbatim} % adds environment for commenting out blocks of text & for better verbatim
\usepackage{subfig} % make it possible to include more than one captioned figure/table in a single float
% These packages are all incorporated in the memoir class to one degree or another...

\usepackage{amsmath}
\usepackage{amssymb}
\usepackage{amsthm}
\usepackage{tikz}
\usepackage{multicol}
\usetikzlibrary{arrows}
\usepackage{hyperref}

\usepackage{geometry}
\geometry{hmargin=2.5cm,vmargin=1.5cm}

\usetikzlibrary{arrows.meta}

 

%%% HEADERS & FOOTERS
\usepackage{fancyhdr} % This should be set AFTER setting up the page geometry
\pagestyle{fancy} % options: empty , plain , fancy
\renewcommand{\headrulewidth}{0pt} % customise the layout...
\lhead{}\chead{}\rhead{}
\lfoot{}\cfoot{\thepage}\rfoot{}

%%% SECTION TITLE APPEARANCE
\usepackage{sectsty}
\allsectionsfont{\sffamily\mdseries\upshape} % (See the fntguide.pdf for font help)
% (This matches ConTeXt defaults)

%%% ToC (table of contents) APPEARANCE
\usepackage[nottoc,notlof,notlot]{tocbibind} % Put the bibliography in the ToC
\usepackage[titles,subfigure]{tocloft} % Alter the style of the Table of Contents
\renewcommand{\cftsecfont}{\rmfamily\mdseries\upshape}
\renewcommand{\cftsecpagefont}{\rmfamily\mdseries\upshape} % No bold!


\theoremstyle{definition}
\newtheorem{exercice}{Exercice}
\newtheorem*{correction*}{Correction}


%%% END Article customizations

%%% The "real" document content comes below...

\title{TP 11 -- Récursivité  -- Algorithmique}
\author{Dr M. Guedj}
\date{} % Activate to display a given date or no date (if empty),
         % otherwise the current date is printed 

\begin{document}
\maketitle


\begin{exercice}[Suite arithmétique]
$\\$
Implantez la suite artihmétique $u$ définie comme suit :
$$
u_{n}=
\left\{
\begin{array}{ll}
	 6			& \text{ si }n= 0\\
	 u_{n-1} + 3 	& \text{ sinon}
\end{array}
\right.
$$
\end{exercice}

\begin{exercice}[Suite arithmétique]
$\\$
Implantez la suite artihmétique $v$ définie comme suit :
$$
v_{n}=
\left\{
\begin{array}{ll}
 	20 			&\text{ si }n= 0\\
 	v_{n-1} + 5 		&\text{ sinon}
\end{array}
\right.
$$
\end{exercice}

\begin{exercice}[Suite géométrique]
$\\$
Implantez la suite géométrique $w$ définie comme suit :
$$
w_{n}=
\left\{
\begin{array}{ll}
 	5000			&\text{ si }n= 0\\
 	w_{n-1}\times 2	&\text{ sinon}
\end{array}
\right.
$$
\end{exercice}

\begin{exercice}[Suite géométrique]
$\\$
Implantez la suite géométrique $w$ définie comme suit :
$$
x_{n}=
\left\{
\begin{array}{ll}
	2				&\text{ si }n= 0\\
	 x_{n-1} \times 0,3		&\text{ sinon}
\end{array}
\right.
$$
\end{exercice}

\begin{exercice}[Factorielle]
$\\$
Implantez, récursivement, la factorielle définie par :  
$$
n!=
\left\{
\begin{array}{ll}
	1			&\text{ si }n= 0\\
	 n\times (n-1)!	&\text{ sinon}
\end{array}
\right.
$$
\end{exercice}

\begin{exercice}[Suite de Fibonacci]
$\\$
Implantez, récursivement, la suite de Fibonacci définie par : 
$$
F(n)=
\left\{
\begin{array}{ll}
	 0			&\text{ si }n= 0\\
	 1			&\text{ si }n= 1\\
	 F(n-1)+F(n-2)	&\text{ sinon}
\end{array}
\right.
$$
Les premiers termes de la suite de Fibonacci sont : 0, 1, 1, 2, 3, 5, 8, 13 et 21.
\end{exercice}

\begin{exercice}[Successeurs]
$\\$
Implantez, récursivement, la suite $succ_{n\in\mathbb{N}}$ définies formellement par : 
$$
succ(n)=
\left\{
\begin{array}{ll}
	 1			&\text{ si }n= 0\\
	 succ(n-1)+1		&\text{ sinon}
\end{array}
\right.
$$
\end{exercice}

\begin{exercice}[Entiers naturels en notation bâton]
$\\$
En notation bâton :
$0$ se note "", 
$1$ se note "I" ,
$2$ se note "II" ,
$3$ se note "III" , ...

Formellement, cette suite est définie par la relation de récurrence suivante : 
$$
\forall n\in\mathbb{N}, baton(n) = 
\left\{
\begin{array}{ll}
	 "" 				&\text{ si }n= 0\\
	 "I"\cdot baton(n-1)		&\text{ sinon}
\end{array}
\right.
$$
où « $\cdot$ » est la relation de concatenation sur les mots ("un"."exemple" = "unexemple").
$\\$
Implantez la fonction \verb|baton()| qui prend en argument un entier et retourne 
sa notation bâton.
\end{exercice}

\begin{exercice}[Puissance]
$\\$
Implantez, récursivement, la puissance $n$-ième ($n\in\mathbb{N}$) du nombre $a\in\mathbb{R}$, 
en utilisant la relation : 
$a^{n}=a.a^{n-1}$.
\end{exercice}

\begin{exercice}[Somme]
$\\$
Implantez, récursivement, la somme des $n$ premiers entiers
(somme(5) = 0+1+2+3+4+5).
\end{exercice}


\end{document}
